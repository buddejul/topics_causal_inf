\documentclass[11pt, a4paper, leqno]{article}
\usepackage{a4wide}
\usepackage[T1]{fontenc}
\usepackage[utf8]{inputenc}
\usepackage{float, afterpage, rotating, graphicx}
\usepackage{epstopdf}
\usepackage{longtable, booktabs, tabularx}
\usepackage{fancyvrb, moreverb, relsize}
\usepackage{eurosym, calc}
% \usepackage{chngcntr}
\usepackage{amsmath, amssymb, amsfonts, amsthm, bm}
\usepackage{caption}
\usepackage{mdwlist}
\usepackage{xfrac}
\usepackage{setspace}
\usepackage[dvipsnames]{xcolor}
\usepackage{subcaption}
\usepackage{minibox}
% \usepackage{pdf14} % Enable for Manuscriptcentral -- can't handle pdf 1.5
% \usepackage{endfloat} % Enable to move tables / figures to the end. Useful for some
% submissions.
\DeclareMathOperator{\E}{E}

\usepackage[
    natbib=true,
    bibencoding=inputenc,
    bibstyle=authoryear-ibid,
    citestyle=authoryear-comp,
    maxcitenames=3,
    maxbibnames=10,
    useprefix=false,
    sortcites=true,
    backend=biber
]{biblatex}
\AtBeginDocument{\toggletrue{blx@useprefix}}
\AtBeginBibliography{\togglefalse{blx@useprefix}}
\setlength{\bibitemsep}{1.5ex}
\addbibresource{refs.bib}

\usepackage[unicode=true]{hyperref}
\hypersetup{
    colorlinks=true,
    linkcolor=black,
    anchorcolor=black,
    citecolor=NavyBlue,
    filecolor=black,
    menucolor=black,
    runcolor=black,
    urlcolor=NavyBlue
}


\widowpenalty=10000
\clubpenalty=10000

\setlength{\parskip}{1ex}
\setlength{\parindent}{0ex}
\setstretch{1.5}


\begin{document}

\title{Topics Causal Inference}
\thanks{Bonn University. Email: s93jbudd[at]uni-bonn.de}

\author{Julian Budde}

\date{
    \today
}

\maketitle

\begin{abstract}
    This term paper presents simulation results for two connected approaches to estimating conditional average treatment effects (CATE):
    The \textit{causal forest} approach originally proposed in \cite{wager2018estimation} and the \textit{generic machine learning} approach discussed in \cite{chernozhukov2023genml}.
    In a first part, I replicate simulation results for causal forest reported in the original paper that demonstrate the ability to estimate a sparse CATE function and perform valid inference.
    I then continue to extend their analysis by lowering the degree of sparseness of the DGP and show that their method breaks down: RMSE is significantly increased and coverage is far below the nominal level.
    I show that inference on the BLP is still possible using the generic machine learning approach. However, for the DGP I consider the RMSE is still considerably worse, illustrating a trade-off between estimation properties and informativeness of the inferential target.
    As an additional point of interest I consider the design of high-dimensional monte carlo studies using the approach suggested by \cite{athey2024wgan}.
\end{abstract}

\clearpage

\section{Introduction}

\section{Methods}

\subsection{Problem Setup}
Throughout we consider a setting with a binary treatment denoted by $D$ taking on values in $\{0, 1\}$.
$Z$ will denote a vector of covariates that has dimension $p$. The sample size will be noted by $n$.
We will consider generally consider a setting of \textit{unconfoundedness}, meaning that conditional on covariates the treatment is as-good-as randomly assigned.
Denoting the outcomes obtained in states of the world with treatment realization $D=d$ by $Y(d)$, this means
\begin{equation*}
    Y(0), Y(1) \perp D | Z.
\end{equation*}
In the context of \cite{chernozhukov2023genml} we will additionally make the assumption that the \textit{propensity score} $p(z) \equiv \Pr(D=1|Z=z)$ is a known function.
The leading case of this is a stratified experiment, with $p(z) = 0.5$ being the special case of a completely randomized experiment.
Generally, the literature also often allows for an unknown propensity score which poses the problem of --- high-dimensional --- first-step estimation.
Note this is not the ``dimensionality'' issue considered here, which is about the complexity of conditional average treatment effects.
The target of this literature is the \textit{conditional average treatment effect} (CATE), which is the average treatment effect for individuals with covariates $Z=z$:
\begin{equation*}
    CATE(Z) = s_0(Z) = \E[Y(1) - Y(0)|Z].
\end{equation*}
Our goal is to estimate and perform inference on this quantity.

Ideally, we would like to have a procedure with the following \textit{properties}:
\begin{itemize}
    \item[(1)] Consistent estimation in the sense that the mean-squared error (MSE) converges to 0;
    \item[(2)] valid inference, i.e. confidence intervals, for $CATE(z)$;
    \item[(3)] ability to adapt to high-dimensional covariate spaces (regimes where $p$ grows with $n$) without making sparsity assumptions;
    \item[(4)] no functional form restrictions on $s_0(z)$.
\end{itemize}
However, we know from general results on optimal achievable non-parametric rates that such a procedure cannot exist.
All proposed approaches then proceed by relaxing one or more of the requirements stated above. The following briefly summarizes some of the ideas, a more comprehensive list of reference can be found in \cite{causalml2024}.
I focus in particular on sparsity assumptions and approximations to the CATE function, because these are two ideas that are taken up in the generic machine learning approach discussed later.
\paragraph*{Sparsity}
A first strategy is to place sparsity assumptions on the CATE function.
For example, a common set of approaches postulates a linear conditional mean function:
\begin{equation*}
    \E[Y|Z] = \sum_{l=1}^{p_1}\alpha_l DX_l + \sum_{j=1}^{p_2}\beta_j W_j,
\end{equation*}
where the covariates $Z$ are split into those determining systematic heterogeneity of treatment effects, $X$, and controls, $W$.
In this model we have hence have $s_0(Z) = S_0(X) = \alpha'X$ under randomization. In these setups we can allow for both $p_1$ and $p_2$ to grow with the sample size.
Hence, the CATE function can be high-dimensional at least for a subset of the covariates.
For these linear models Lasso-type approaches adapted to causal inference (so called "double" lasso) will perform well, under a condition called \textit{approximate sparsity}.
Intuitively, this condition is fulfilled by DGPs where the magnitude of the (ordered) coefficients $\alpha_l, \beta_l$, decays sufficiently fast.
One such DGP is for example given by
\begin{equation*}
    \alpha_l = \frac{1}{l^2}, \qquad \beta_j = \frac{1}{j^2}.
\end{equation*}
Assuming $p_1$, the dimension of heterogeneity, does not grow too fast and we have approximate sparsity, it is possible to show an asymptotic normal approximation, namely
\begin{equation*}
    \sqrt{n}(\hat{\alpha}-\alpha) \sim_a N(0,V)
\end{equation*}
where $V$ can be consistently estimated. These results are shown in \cite{belloni2015uniform} and \cite{belloni2018uniformly}.

\paragraph*{Approximations to the CATE}
Another approach is taken in \cite{semenova2021debiased}:
They propose to focus on a \textit{low-dimensional} subset $X$ of the covariates $Z$, which the researcher has to decide on ex ante.
Then, the new target for inference becomes the \textit{partial} CATE $s_0(X)$.
Their paper then simultaneously allows for \textit{unconfoundedness} which poses the challenge of estimating $p(Z)$ --- potentially using machine learning methods --- in a first step.
In a second step, they the nfocus on a \textit{linear} approximation to the (partial) CATE.
Hence, they essentially use the following approximations to make the problem solvable:
\begin{equation*}
    s_0(Z) \approx s_0(X) \approx \beta'X.
\end{equation*}
The best linear predictor (BLP) problem in this case is given by
\begin{equation*}
    \min_{b\in\mathbb{R}^p} \E [(s_o(X) - X'b)^2].
\end{equation*}
We can then make use of the following identification result (which is sometimes referred to as the Horvitz-Thomson transform):
\begin{equation*}
    s_0(Z) = \E[YH|Z]
\end{equation*}
where $H = \frac{D-p(Z)}{p(Z)(1-p(Z))}$.
It is then easy to derive the solution of the BLP problem above as
\begin{equation*}
    \beta = E[XX']^{-1}E[XHY]
\end{equation*}
which instead of the BLP of $Y$ uses the transformed outcome $HY$.
For this we can then make use of standard least squares theory --- given that the nuisance parameter estimation in the first step satisfies certain regularity conditions --- and show
\begin{equation*}
    \hat{\beta} - \beta_0 \sim_a N(0, \Omega/N).
\end{equation*}
\subsection{Generic Machine Learning}

\section{Simulation and Results}




\clearpage
\newpage

% \setstretch{1}
\printbibliography
\setstretch{1.5}

% \appendix

% The chngctr package is needed for the following lines.
% \counterwithin{table}{section}
% \counterwithin{figure}{section}

\end{document}
